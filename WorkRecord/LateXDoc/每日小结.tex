% 文档格式:article+中文
\documentclass[12pt,a4paper]{ctexart}
% 设置页眉页脚
\usepackage{fancyhdr}
\pagestyle{fancy}
% 右页眉不显示
\fancyhead[R]{}
% 中间页脚显示页码
\fancyfoot[C]{\thepage} 
% 页眉分隔线宽度4磅
\renewcommand{\headrulewidth}{4pt} 
% 长注释 \iffalse 设置超链接 \fi
\usepackage{hyperref}
\usepackage{ulem}
\hypersetup{
	colorlinks=true,
	linkcolor=blue,
	filecolor=magenta,      
	urlcolor=cyan,
	pdftitle={Work Abstract},
	pdfpagemode=FullScreen,
}
\urlstyle{same}

\title{\heiti 2022工作日报}
\author{Cuiyang}
\date{\kaishu 2022年9月27日}

\begin{document}
	\maketitle
	\tableofcontents
	\clearpage
	\section{2020}
		\subsection{待补充}
	% \clearpage
	\section{2021}
		\subsection{待补充}
	% \clearpage
	\section{2022}
		\subsection{五月版}
	\begin{itemize}
		\item 开发ASR+讯飞SDK
		\begin{itemize}
			\item 写Presenter+Logic结构
		\end{itemize}
		\item 长链接,待测试
		\item 长链接:地址是否有问题?生产的语音界面
		\item 长链接:异常处理
		\item 分享微信文件到新Call浦
		\item 拉广告页失败,改连接
		\item 分栏优化:1⃣️更改Label的存储逻辑2⃣️不展示Label时置顶收缩出问题
		\item Camera2调研
		\item 搜索栏增加麦克风
		\item 有人在呼叫的时候,其他人回到主界面悬浮窗不会展示
		\item 修改创建群聊接口///不需要finish,群名
		\item 修改创建群聊同人不同名,需要创建新需求—根据需求单进行建群,有同人可能。
		\item 飞云流水线重新搭建———
		
		\item 1. 调整label字体大小,发视频
		\item 2. ASR“打开智慧办公”发送两次BUG修改
		\item 评估SIM需求
		\item 评估微信通讯需求———需要用到端上聊天能力,不建议
		\item Camera2使用Demo做一个———放弃,前端做
		\item WS: 录音超时时间:2min;并发量过大提示语
		\item 居老师发现的小问题,待修改确认
		\item 建群发现BUG:拉人失败问题———findExitGroup接口问题
		\item Dev通讯录问题——建华追踪-解决
		
	\end{itemize}

	\newpage
	\subsection{七月版}
	\begin{itemize}
		\item 0525: IM配置问题
		\item 0525: ASR弹框频繁+onFailure反复回调
		
		\item 0607:BUG—会话崩溃,ConcurrentModification
		\item 0607:BUG—类型强转错误
		\item 0607:BUG—空指针问题
		\item 0607:飞云制品库—做不了,需要网管开权限。
		
		\item 0609:问卷调研回访
		\item 0609:enum空指针
		
		\item 0613: 1100-ASR生产链接更换——代码待合并
		\item 0613: 1100-HiAppActivity去除标题的设置级别判断——已修复,待测试合并。
		\item 0617: ASR前端更改跳转,android出现问题,改————已修复,待测试合并。
		\item ————————————
		\item 0622: 新版令牌开发——PRD已出—UI完成开发
		\item 0614-15: 追踪BUG,切换fragment异常。
		\begin{itemize}
			\item 采用埋日志方式,等下次发现
		\end{itemize}
		\item 0628-ASR:bug追踪——
		\item 0627-0701:开发令牌
		\item 0628-开发JSSDK:调用双击刷新
		\item 0629:评估开发返回主界面刷新界面的JSSDK-----已有接口
		\item 0701:ASR申请权限BUG
		\item 0705:软件闪退,修复空指针BUG---ConversationListAdapter
		\item 0705、0708:开发群机器人自动添加JSSDK———进度:获取app列表为空
		\item 0706:测试宁盾令牌,开发令牌设置界面
		\begin{itemize}
			\item 修改包名解决问题Base64
			\item Base.decode使用参数SAFE\_MODE
		\end{itemize}
		\item 0706:打包测试ASR,最终生产
		\item 0708:修复扫码BUG,连接问题
		\item 0715:Asr首屏问题修复
		\item 0715:BUG会话列表刷新问题(尝试关闭动画看能否解决)
		\item 0715:BUG会话分类取消后默认显示全部聊天
		\item 0705、0718、0725:BUG追踪复现的切换Tab问题
		\newline
		\item 0715:BUG全员呼叫“邀请成员失败”
		\begin{itemize}
			\item 0715: com.baidu.hi.voice.interactor.CallInteractor\#onAddMemberResponse日志
		\end{itemize}
		\item 0718:修改JSSDK添加机器人接口
		\item 0719:开发手机令牌,完善逻辑——扫码崩溃和相关逻辑
		\item 0721:手机令牌存储逻辑大改
		\item 0721:JSSDK添加机器人接口问题———重复调用
		\item 0721:JSSDK子项目:修改建群接口调用方式—-全局变量在sendCreateGroupReq中赋值
		\item 0722:制定折叠重构,setExpandItem函数和isExpand变量的调用逻辑
		\item 0725、0727:重写添加机器人方案
		\item 0728:修改令牌刷新和弹框
		\item 0802:开发添加好友发送默认消息
		\item 0803:调研优化工作站展示的UI
		\item 0810:手机令牌-SM3加密工号,进行手机令牌鉴权SM3Util.hash()
		\item 0826:手机令牌加灰度(CopyUtil)
		\item 0826:UI—修改各种size+ 英文化
		\item 0826:清空令牌数据后的跳转方式
	\end{itemize}
	
	\newpage
	\subsection{九月版}
	\begin{itemize}
		\item 0810-11:工作站改版-UI开发
		\item 0812-工作站开发进度:开发Tab切换动画,如何恢复成原来的Text
		\begin{itemize}
			\item 参考:https://www.likecs.com/show-205250598.html
		\end{itemize}
		\item 点击动画做好后,进行数据展示的处理。
		\item 0818:完成基本展示
		\item 0818:添加“空数据”Item-同之前的空界面
		\item 0818:移除“添加”Item
		\item 0819:过滤展示的数据
		\item 0819:TabLayout滚动效果,初始化后自动滚动到中间—每次进入重制
		\item 0819:点击“确定”后,常用和全部应用中的数据差异计算和提交
		\item 0818:设置拖动事件
		\item 0819:搜索添加常用应用后,底部全部应用同步更新
		\item 0819:删除常用应用后,底部全部应用同步更新
		\item 0823:清空常用应用出问题,最后一个无法删除。。。
		\item 0823:最后一个应用删除,然后取消,下面应用变多了。
		\begin{itemize}
			\item BUG:状态一致问题
		\end{itemize}
		\item 0822:编辑模式切换Tab后,”退出”编辑模式,需要将缓存数据更新(presenter.getBeforeEditEapps())
		\item 0822:点击绿色按钮添加应用到常用,状态同步问题。
		\item 0822:“撤销删除常用”应用,点击事件处理缓存——0826:BUG修复,撤销后应用数据刷新错误
		\item 0823:点击添加和删除,更新服务端和数据库数据
		\item 0824:编辑模式过滤没用的应用
		\item 0824:点击空应用的按钮“添加”,设置编辑模式,而非跳转
		\item 0824:BUG:ConcurrentModificationException
		\item 0824:清空数据重新登录看是否保持应用数量!!
		\item 0824:多点几次,出现单个绿色应用在常用上
		\item 0826:常用应用拖动排序问题
		\item 0826:刷新后数据TabPosition更新错误!!!
		\item 0829:搜索添加数据位置异常修复BUG
		\item 0830:每次刷新数据后记得刷新全部Tab(onResume刷新缓存数据有问题,在onCreate刷新)
		\item 0901:拖动后出现绿色按钮。。。。刷新问题——clearView
		\item 0901:修改空白数据提示文字
		\item 0902:ListItem修改为自适应高度
		
		\item 0902:整理1200代码。
	\end{itemize}
	—————————————————
	\newpage
	\subsection{十月版}
	\begin{itemize}
		\item 0905-0909:工作站相关刷新BUG、空分类过滤UI展示等
		\item 0906:手机令牌灰度添加和测试,修改刷新按钮方式
		\item 0919:查看文档和prd,联系相关人员讨论SIM卡需求,申请账号等
		\item 0921:Kotlin及联系SIM卡需求
		\item 0926:未读消息计算错误BUG,服务端下发问题
		\newline get\_list\_change协议
		\item 0927:配置Charles抓包+Git复习
		\item 0928:协助进行会话未读数BUG抓包
		\item 0929:开发英文化
		\item 0929-0930:开发英文化
		\item 0930:Callpu名称更改
		
	\end{itemize}
	-----------------------------------
	\newpage
	\subsection{十二月}
	\begin{itemize}
		\item 1009:英文化
		\item 1011 :优化会话配置拖动效果
		\item 1012: SIM卡认证对接
		\begin{itemize}
			\item[-] 新项目/沿用浦银接口:网银的项目下子项目,沿用接口。
			\item[-] 1018、1026:开发界面功能
			\item[-] 待定:逻辑开发(后台依赖)
		\end{itemize}
		\item 1013:追踪新建环境登录失败的BUG
		\begin{itemize}
			\item[-] 持续跟进,后台服务问题,TCP登录模块不明原因导致超时
			\item[-] 1018:协议 ser\_service,升级到百度追查
		\end{itemize}
		\item 1014:需求-Eapp添加英文化字段(数据库升级逻辑)
		\begin{itemize}
			\item[-] 1031:数据库升级逻辑追踪
			\item[-] 1028:企业推荐应用英文化Eapp
			\item[-] 1028:群应用英文化GroupApp
			\item[-] 1103:数据库升级逻辑处理,\underline{\textbf{覆盖安装升级数据库}}
		\end{itemize}
		\item 1014:联系群应用英文化接口
		\begin{itemize}
			\item[-] 涉及连接:https://ip:port/xpc/
			\item[1.] 群机器人接口:group/applist?v=2,group/appget
			\item[2.] Eapp机器人 v3接口:app/ids,app/get
		\end{itemize}
		\item 1014:GitHub旧项目从https鉴权切换为ssh鉴权
		\begin{itemize}
			\item[-] \underline{\href{https://blog.csdn.net/weixin_43065003/article/details/126973140}{参考博客(点击跳转)}}
			\item[1.] git remote -v
			\item[2.] git remote set-url origin git@github.com:baizhu0414/ProName.git
		\end{itemize}
		\item 1018:抓包HF环境,登录ser\_service出错
		\item 1020:ASR识别不出来BUG- hi\_websocket=1协议错
		\item 1025:排查日程通知BUG- push\_msg\_notify协议错,是否与推送二选一?
		\item 1028:整理动态口令代码,临时打包--去除冗余配置页xml
		\item 1031:百度BdExplorer内核替换—查找可能出现的BUG(待办)
		\item 1031:集成埋点依赖(独立于现有埋点体系)
		\begin{itemize}
			\item[-] 1103:查看火眼SDK继承需求,评估工作量
			\item[-] 1107:沟通,获取SDK,注册参数
			\item[-] 1109:开发测试火眼监测埋点
			\item[-] 1110:查看Flavor的方法(多渠道)-BuildConfig里面有
			\item[-] 1114,1115,1116:打通火眼SDK,现在连不通。
			\item[-] 1114,1118:构建单例类,抽离代码,代码规范化:
			\item[-] -  细化分类:PageType + EventType
			\item[-] 1122:梳理埋点的代码结构
		\end{itemize}
		\item 1103:查看PublicCalendar-BUG-推送协议JSON错误
		\item 1108:HF登录环境排查—publicKey请求超时app4地址
		\item 1109:群机器人自动添加失败BUG——appGet接口405查询失败
		\item 1109:修复工作站空指针崩溃BUG
		\item 1114:测试1200版本功能用例
		\newpage
		\item 1115:
		\begin{itemize}
			\item[-] 机器人查询为空抓包—机器人不存在
			\item[-] 工作站点击上报失败问题排查(泪目,上传了,但是文件夹不一样)
		\end{itemize}
		\item 1121:BUG-置顶折叠8变成7条,在小屏手机上没有让折叠Item消失
		\item 1121:
		\begin{itemize}
			\item[-] BUG-HF下发HiDns地址问题
			\item[-] 1121发现,1123:BUG-置顶折叠8变成7条,在小屏手机上没有让折叠Item消失
		\end{itemize}
		\item 1130: 重新进行修复,梳理逻辑,并进行多机型测试。
		\item 1125: 
		\begin{itemize}
			\item[-] BUG-搜索文件互传助手
			\item[-] 联系人分类助手
			\begin{sloppypar}
				\item[-] 点击富文本出问题是因为消息类型问题,com.baidu.hi.entity.ChatInformation.UrlParamsHolder\#URL\_TYPE\_EVENT这个类型发送com.baidu.hi.bean.command.PublicTextLinkCommand没处理,所以跳转不了。应该是开发的时候处理丢了吧?
			\end{sloppypar}
				\item 1207:记住密码后登录的流程$\backslash$EAP相关链接追踪
				\begin{sloppypar}
					\item[-]  com.baidu.hi.database.FriendsDBUtil\#\\GLOBAL\_SEARCH\_EMP\_SQL\_WITH\_SEARCH\_FEATURE
				\end{sloppypar}
			\item[-] Eapp分类助手
			\item[-] 1130: 联系人可查到
			\item[-] 1201: 查询Eapp
		\end{itemize}
		\item 1208:BUG-超文本中链接点击失败
		\item 1212-1214:已完成需求代码梳理
		\item 1215:抓包-
		\begin{itemize}
			\item[-] 群投票
			\item[-] 工作站会议请求等
		\end{itemize}
		\item 1215:梳理火眼代码
		\item 1216:梳理卡片消息类型的流程
		\item 1229-0103: 追踪开启直播失败BUG———发现一处单词拼写错误导致JSON可能转换失败
		\section{2023}
		\subsection{一月}
		\item 0103: 协助抓包群投票
		\item 0105:
		\begin{itemize}
			\item[-] 协助查看isNative错误问题
			\item[-] ~0106协助查看showTeller在uatRelease包和uatDebug包不同表现问题
			\item[-] 修改动态口令BUG
		\end{itemize}
		\item 0109:继续查看uatRelease包的showTeller,确认整体流程是否有问题
		\item[--] showTellerSDK也有BUG,一起更新。
		\item 0109:追踪手势密码登录参数,后端需要写压测脚本
		\newpage
		\item 0110:
		\begin{itemize}
			\item[-] 协助抓包:TCP+ 匿名登陆
			\item[-] 匿名登陆流程追踪
		\end{itemize}
		\item 0113:追踪分享云盘文件变成回执消息流程
		\item 0118:配合抓包-
		\begin{itemize}
			\item[-] Eap-cas-person
			\item[-] REAL环境
		\end{itemize}
		
		\item 0119:处理动态口令不显示问题
		\begin{itemize}
			\item[-] 协助服务端处理动态口令登录问题
		\end{itemize}
		\item 0119:更新最新版AS+jdk11
		\begin{itemize}
			\item[-] 运行一个git项目
		\end{itemize}
		\emph{\large{\textbf {新春快乐~}}}
		\item  0201: 测试Pad是否有问题
		\item 0202: 重构EappRecyclerView:
		\begin{itemize}
			\item[-] 拖动后的recyclerView UI刷新优化,全剧刷新卡顿,改为局部刷新。
			\item[-] 0203:BUG:点击过快可能导致数据错乱,使用HashSet替代ArrayList。
			\item[-] BUG待修复:Eapp有时候会自己清空———没有复现成功
		\end{itemize}
		\item 0202: 处理版本更新通知事项
		
		\item 0203\textbackslash0206: 查看项目中需要英文化的图片(ios、android需要对齐)
		\item 0206: 将英文化字符串替换
		\item 0206:协助后台排查REAL登录失败问题:publicKey公钥获取失败
		\item 0207:review火眼sdk代码,整理完善patch。
		\item 0209:给工作站添加数据更新日志,便于BUG排查
		\item 0210:重构火眼SDK代码,在LogReport相关位置上传。
		\begin{itemize}
			\item[-] 0214:详细梳理日志上报,根据已有埋点进行上传。
			\item[-] 0215:火眼SDK突然卡顿,发送日志给供应商。
		\end{itemize}
		\item 0210:整理英文化代码—群应用,工作站,公众号(PulbicActount)
		\item 0208:协助后台排查安卓Pad登录手势密码多次设置问题。
		\item 0213:排查数组相关BUG(可能数组越界查过int byte数组大小2G=2\^31-1 )
		\item 0215:成功复现并彻底解决0202发现的工作站清空BUG
		\item 0216:英文化图片——UI要求—追踪应用场景
		\item 0220-0222:火眼上传数据有问题,协助sdk提供方排查结果
		\item 0220-0221:协助抓日志排查直播失败BUG
		\item 0222:抓包建群和开启语音
		\item 0223:何业洋转存选项没有
		\item 0223:win端和android端置顶数量不同步———没查出问题
		\item 0223:客户经理sdk登陆问题排查
		\item 0227:加固release排查转存BUG
		\item 0225:火眼sdk日志混乱,尝试更新
		\item 0228:一键建群有时失败,查看日志。
		\item 0301:初步查明长按消息弹框选项不全的BUG,数据库获取Eapp属性不正确。
		\item 0302:app/get打日志继续查看是否“更新”数据正常。
		\item 0314:修复长按聊天文件长按弹框不完全问题。
		\item 0301:下载虚拟机,尝试安装Call浦。
		\item 0301:调研Call浦检测虚拟机的技术。
		\item 0301:调研目前前端界面操作可选项,后面改成服务端全局控制是否展示。
		\item HiAppActivity\#createMenu(int, java.util.List<MenuItemFastJson>, android.util.SparseIntArray)
		\item 0317:开发控制功能,超链接参数控制部分按钮显示。
		\item 0302-0303:整理代码,参与review,并且进行相应修改。
		\item 0308:打包加固
		\item 0309:调研北斗集成方法,没找到具体方案。硬件相关。
		\item 0316:完成高德地图集成和地址转换,底层使用北斗导航。
		\item 0313:督促行员升级软件
		\item 0321-0322:火眼生产环境测试——sdk需要修改
		\item 0322: 排查李研敏的安卓置顶聊天小时BUG,根据现象,预计是数据库相关的逻辑问题。
		\item 0322:排除数据库问题,可能是后端数据问题。还需要排查oppositeUid相关判断逻辑。
		\item 0324:怀疑是headerType问题,仍旧是后端问题。添加日志,方便后期发现排查。
		\item 0322:修复动态口令又是网络失败不展示BUG,设置为true
		\item 0322:查看语音识别JSSDK接口,完善接口文档。
		\item 0327:查看添加常用应用问题—行长BUG-------没有找到BUG位置
		\item 测试其他数据库操作+完善数据库close操作逻辑
		\item 0327:数据库升级
		\item 数据库关闭方法和逻辑:以比较好的方法关闭数据库,可以节省资源,同时提高操作效率。(此处不用缓存的情况)
		\item SQLiteOpenHelper会自动关闭,只需关闭cursor即可。
		\item 完善文件相关操作‘
		\begin{itemize}
			\item[-] initFolder等HiApplication功能
			\item[-] 0329:查看日志打印逻辑
			\item[-] 0406:崩溃日志打印
			\item[-] 0407:开发权限申请逻辑
			\item[-] 0424-0523:开发ActionBar,Navigator组件
			\item[-] 0412:开发图片切割操作
			\item[-] 0412:开发简单的消息总线,分析OttoEvent
			\item[-] 0412:开发简单Http工具,搞个服务器尝试部署各种服务。
		\end{itemize}
		\item 0328:协助家俊追踪长按打印文件的参数处理
		\begin{itemize}
			\item[-] ResponseCallback\#onSuccess
		\end{itemize}
		
		\item 0330:英文化Prd细节查看和评论
		\item 0331:Pad横屏
		\item 0331:虚拟机检测
		\begin{itemize}
			\item SpdbUtils\#getRemoteUrlWithKey
			\item NativeWebViewModule\#getURL
		\end{itemize}
		\item 0331:高德地图
		\item 0331,0404:更新火眼SDK
		\item 0403:加固软件
		\item 0404,0406:查看英文化文档,是否有后端内容
		\item 0412: 排查打卡定位问题
		\item 0414:排查没有安装软件权限时升级软件
		\item 0414、0419:研究打包机接入“飞云”安全检测网站
		\item 0418,0424:英文化开发(IM、应用等),测试文档完善
		% 中划线
		\item \sout{0419:省电功能}
		\item 0420:协助打包、抓包等
		\item 0427:整理登陆后http请求
		\item 0505:IM英文化测试打包——uat
		\item 服务号英文化开发——dev
		\item 0505: 应用英文化开发测试——延期uat
		\item 0509:后端修改,代码修改
		\item 服务端下发会议分享信息的英文化——前端界面请求后台,无关内容
		\item 0506:驾驶模式英文图片更改
		\item \sout{灰度显示动态口令的onFailure改一下}
		\item msg\_notify报文解析查看,工作流-直播\/智能记录通知
		\item \sout{‘收藏’的html文件}
		\item 0509:下发通知英文化:
		\begin{itemize}
			\item[-] 1. 日程:“远程会议”英文化;
			\item[-] 0510:测试时间显示,’无主题’英化
			\item[-]  2. 工作流:英文化。ChatListitemPublicNews
		\end{itemize}
		\item 协助打包加固工作:produatrelease,1301加固
		\item 聊天分类拖动超过“全部”然后再点击灰色按钮,导致错乱
		\item 更新火焰3.0.5版本
		\item BUG:推送消息不在首页显示
		\item BUG:   英文化服务号在首页conversation表的英文化,需要新增msg\_body\_en字段
		\newline
	\end{itemize}
\end{document}